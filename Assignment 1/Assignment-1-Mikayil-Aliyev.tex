% Options for packages loaded elsewhere
\PassOptionsToPackage{unicode}{hyperref}
\PassOptionsToPackage{hyphens}{url}
%
\documentclass[
]{article}
\usepackage{amsmath,amssymb}
\usepackage{lmodern}
\usepackage{ifxetex,ifluatex}
\ifnum 0\ifxetex 1\fi\ifluatex 1\fi=0 % if pdftex
  \usepackage[T1]{fontenc}
  \usepackage[utf8]{inputenc}
  \usepackage{textcomp} % provide euro and other symbols
\else % if luatex or xetex
  \usepackage{unicode-math}
  \defaultfontfeatures{Scale=MatchLowercase}
  \defaultfontfeatures[\rmfamily]{Ligatures=TeX,Scale=1}
\fi
% Use upquote if available, for straight quotes in verbatim environments
\IfFileExists{upquote.sty}{\usepackage{upquote}}{}
\IfFileExists{microtype.sty}{% use microtype if available
  \usepackage[]{microtype}
  \UseMicrotypeSet[protrusion]{basicmath} % disable protrusion for tt fonts
}{}
\makeatletter
\@ifundefined{KOMAClassName}{% if non-KOMA class
  \IfFileExists{parskip.sty}{%
    \usepackage{parskip}
  }{% else
    \setlength{\parindent}{0pt}
    \setlength{\parskip}{6pt plus 2pt minus 1pt}}
}{% if KOMA class
  \KOMAoptions{parskip=half}}
\makeatother
\usepackage{xcolor}
\IfFileExists{xurl.sty}{\usepackage{xurl}}{} % add URL line breaks if available
\IfFileExists{bookmark.sty}{\usepackage{bookmark}}{\usepackage{hyperref}}
\hypersetup{
  pdftitle={STAT3064: Assigmment 1},
  hidelinks,
  pdfcreator={LaTeX via pandoc}}
\urlstyle{same} % disable monospaced font for URLs
\usepackage[margin=1in]{geometry}
\usepackage{color}
\usepackage{fancyvrb}
\newcommand{\VerbBar}{|}
\newcommand{\VERB}{\Verb[commandchars=\\\{\}]}
\DefineVerbatimEnvironment{Highlighting}{Verbatim}{commandchars=\\\{\}}
% Add ',fontsize=\small' for more characters per line
\usepackage{framed}
\definecolor{shadecolor}{RGB}{248,248,248}
\newenvironment{Shaded}{\begin{snugshade}}{\end{snugshade}}
\newcommand{\AlertTok}[1]{\textcolor[rgb]{0.94,0.16,0.16}{#1}}
\newcommand{\AnnotationTok}[1]{\textcolor[rgb]{0.56,0.35,0.01}{\textbf{\textit{#1}}}}
\newcommand{\AttributeTok}[1]{\textcolor[rgb]{0.77,0.63,0.00}{#1}}
\newcommand{\BaseNTok}[1]{\textcolor[rgb]{0.00,0.00,0.81}{#1}}
\newcommand{\BuiltInTok}[1]{#1}
\newcommand{\CharTok}[1]{\textcolor[rgb]{0.31,0.60,0.02}{#1}}
\newcommand{\CommentTok}[1]{\textcolor[rgb]{0.56,0.35,0.01}{\textit{#1}}}
\newcommand{\CommentVarTok}[1]{\textcolor[rgb]{0.56,0.35,0.01}{\textbf{\textit{#1}}}}
\newcommand{\ConstantTok}[1]{\textcolor[rgb]{0.00,0.00,0.00}{#1}}
\newcommand{\ControlFlowTok}[1]{\textcolor[rgb]{0.13,0.29,0.53}{\textbf{#1}}}
\newcommand{\DataTypeTok}[1]{\textcolor[rgb]{0.13,0.29,0.53}{#1}}
\newcommand{\DecValTok}[1]{\textcolor[rgb]{0.00,0.00,0.81}{#1}}
\newcommand{\DocumentationTok}[1]{\textcolor[rgb]{0.56,0.35,0.01}{\textbf{\textit{#1}}}}
\newcommand{\ErrorTok}[1]{\textcolor[rgb]{0.64,0.00,0.00}{\textbf{#1}}}
\newcommand{\ExtensionTok}[1]{#1}
\newcommand{\FloatTok}[1]{\textcolor[rgb]{0.00,0.00,0.81}{#1}}
\newcommand{\FunctionTok}[1]{\textcolor[rgb]{0.00,0.00,0.00}{#1}}
\newcommand{\ImportTok}[1]{#1}
\newcommand{\InformationTok}[1]{\textcolor[rgb]{0.56,0.35,0.01}{\textbf{\textit{#1}}}}
\newcommand{\KeywordTok}[1]{\textcolor[rgb]{0.13,0.29,0.53}{\textbf{#1}}}
\newcommand{\NormalTok}[1]{#1}
\newcommand{\OperatorTok}[1]{\textcolor[rgb]{0.81,0.36,0.00}{\textbf{#1}}}
\newcommand{\OtherTok}[1]{\textcolor[rgb]{0.56,0.35,0.01}{#1}}
\newcommand{\PreprocessorTok}[1]{\textcolor[rgb]{0.56,0.35,0.01}{\textit{#1}}}
\newcommand{\RegionMarkerTok}[1]{#1}
\newcommand{\SpecialCharTok}[1]{\textcolor[rgb]{0.00,0.00,0.00}{#1}}
\newcommand{\SpecialStringTok}[1]{\textcolor[rgb]{0.31,0.60,0.02}{#1}}
\newcommand{\StringTok}[1]{\textcolor[rgb]{0.31,0.60,0.02}{#1}}
\newcommand{\VariableTok}[1]{\textcolor[rgb]{0.00,0.00,0.00}{#1}}
\newcommand{\VerbatimStringTok}[1]{\textcolor[rgb]{0.31,0.60,0.02}{#1}}
\newcommand{\WarningTok}[1]{\textcolor[rgb]{0.56,0.35,0.01}{\textbf{\textit{#1}}}}
\usepackage{graphicx}
\makeatletter
\def\maxwidth{\ifdim\Gin@nat@width>\linewidth\linewidth\else\Gin@nat@width\fi}
\def\maxheight{\ifdim\Gin@nat@height>\textheight\textheight\else\Gin@nat@height\fi}
\makeatother
% Scale images if necessary, so that they will not overflow the page
% margins by default, and it is still possible to overwrite the defaults
% using explicit options in \includegraphics[width, height, ...]{}
\setkeys{Gin}{width=\maxwidth,height=\maxheight,keepaspectratio}
% Set default figure placement to htbp
\makeatletter
\def\fps@figure{htbp}
\makeatother
\setlength{\emergencystretch}{3em} % prevent overfull lines
\providecommand{\tightlist}{%
  \setlength{\itemsep}{0pt}\setlength{\parskip}{0pt}}
\setcounter{secnumdepth}{-\maxdimen} % remove section numbering
\ifluatex
  \usepackage{selnolig}  % disable illegal ligatures
\fi

\title{STAT3064: Assigmment 1}
\author{}
\date{\vspace{-2.5em}}

\begin{document}
\maketitle

\begin{enumerate}
\def\labelenumi{\arabic{enumi}.}
\item
  \begin{enumerate}
  \def\labelenumii{\alph{enumii})}
  \item
    Why is it important to describe the mathematical model we want to
    simulate from and why should one not automatically choose the
    Gaussian model for a simulation?

    It is important to describe the mathematical model we want to
    simulate from because we will require to use that model to
    estimate/predict trends and patterns if we find that the model that
    has been described demonstrates simulations which closely match the
    observed experimental results.

    When selecting a mathematical model to perform simulations from, one
    must not automatically choose a Gaussian model to analyse, as it is
    unlikely that all experimental contexts will be producing trends
    which can be accurately described by a Gaussian Model. Examples of
    this may include a growth/decay context which may be better suited
    by a exponential model, or current-day COVID cases being better
    represented by a logistical model.
  \item
    Give two reasons why simulations should be reproducible.

    \begin{enumerate}
    \def\labelenumiii{\arabic{enumiii}.}
    \item
      Investigation / Research may be progeressed in an efficient
      manner, and the reproducibility will allow for the simulation to
      be re-used or compared with different simulations to evaluate
      effectiveness of a model.
    \item
      To check and verify the validty of the simulations by an external
      party. This emerges from having the simulation in a reproducible
      state, with the relevant steps and assumptions documented, which
      can be critically examined to review validity.
    \end{enumerate}
  \item
    Give details of the variance calculation var(\(W_2\)) for Example
    4.2 similar to those for var(\(W_1\)).

    var(\(W_2\)) = var(\(n_2 ^T\)\textbf{\emph{X}}) =
    var(\(n_{21}\)\textbf{X}\(_1\) + \(n_{22}\)\textbf{X}\(_2\))

    = \(n_{21}^2\)var(\textbf{X}\(_1\)) +
    \(n_{22}^2\)var(\textbf{X}\(_2\)) + 2\(n_{21}n_{22}\) *
    cov(\textbf{X}\(_1\), \textbf{X}\(_2\))

    = 0.31\(^2\) * 2.4 + 0.95\(^2\) * 1 + 2 * 0.31 * 0.95 * (-0.5)

    = \textbf{0.84}
  \end{enumerate}
\item
  Consider the aircraft data with the logged variables as in Question 2
  of Computer Lab 1. Divide the data into the three period groups. We
  are interested in comparing changes that occur over time.

  Setup working environment
\end{enumerate}

\begin{Shaded}
\begin{Highlighting}[]
\FunctionTok{rm}\NormalTok{(}\AttributeTok{list =} \FunctionTok{ls}\NormalTok{())}

\ControlFlowTok{if}\NormalTok{ (}\SpecialCharTok{!}\FunctionTok{is.null}\NormalTok{(}\FunctionTok{sessionInfo}\NormalTok{()}\SpecialCharTok{$}\NormalTok{otherPkgs)) \{}
  \FunctionTok{invisible}\NormalTok{(}
    \FunctionTok{lapply}\NormalTok{(}\FunctionTok{paste0}\NormalTok{(}\StringTok{\textquotesingle{}package:\textquotesingle{}}\NormalTok{, }\FunctionTok{names}\NormalTok{(}\FunctionTok{sessionInfo}\NormalTok{()}\SpecialCharTok{$}\NormalTok{otherPkgs)),}
\NormalTok{           detach, }\AttributeTok{character.only=}\ConstantTok{TRUE}\NormalTok{, }\AttributeTok{unload=}\ConstantTok{TRUE}\NormalTok{)}
\NormalTok{  )}
\NormalTok{\}}

\FunctionTok{options}\NormalTok{(}\AttributeTok{stringsAsFactors =} \ConstantTok{FALSE}\NormalTok{)}

\FunctionTok{library}\NormalTok{(ggplot2)}
\FunctionTok{library}\NormalTok{(tidyverse)}
\end{Highlighting}
\end{Shaded}

\begin{verbatim}
## -- Attaching packages --------------------------------------- tidyverse 1.3.1 --
\end{verbatim}

\begin{verbatim}
## v tibble  3.1.3     v dplyr   1.0.7
## v tidyr   1.1.3     v stringr 1.4.0
## v readr   2.0.0     v forcats 0.5.1
## v purrr   0.3.4
\end{verbatim}

\begin{verbatim}
## -- Conflicts ------------------------------------------ tidyverse_conflicts() --
## x dplyr::filter() masks stats::filter()
## x dplyr::lag()    masks stats::lag()
\end{verbatim}

\begin{Shaded}
\begin{Highlighting}[]
\FunctionTok{library}\NormalTok{(MASS)}
\end{Highlighting}
\end{Shaded}

\begin{verbatim}
## 
## Attaching package: 'MASS'
\end{verbatim}

\begin{verbatim}
## The following object is masked from 'package:dplyr':
## 
##     select
\end{verbatim}

\begin{Shaded}
\begin{Highlighting}[]
\FunctionTok{library}\NormalTok{(GGally)}
\end{Highlighting}
\end{Shaded}

\begin{verbatim}
## Registered S3 method overwritten by 'GGally':
##   method from   
##   +.gg   ggplot2
\end{verbatim}

\begin{Shaded}
\begin{Highlighting}[]
\FunctionTok{setwd}\NormalTok{(}\StringTok{\textquotesingle{}C:/Users/aliye/Documents/STAT3064/Assignment 1\textquotesingle{}}\NormalTok{)}

\NormalTok{aircraft }\OtherTok{=} \FunctionTok{read.csv}\NormalTok{(}\StringTok{\textquotesingle{}aircraft.csv\textquotesingle{}}\NormalTok{)}
\end{Highlighting}
\end{Shaded}

\begin{enumerate}
\def\labelenumi{\alph{enumi})}
\tightlist
\item
  Show smoothed histograms of logLength and logPower separately for the
  three periods. Comment on the shapes of the histograms and how the
  change over time affects this shape.
\end{enumerate}

\begin{Shaded}
\begin{Highlighting}[]
\NormalTok{logPower }\OtherTok{=} \FunctionTok{log10}\NormalTok{(aircraft}\SpecialCharTok{$}\NormalTok{Power)}
\NormalTok{logSpan }\OtherTok{=} \FunctionTok{log10}\NormalTok{(aircraft}\SpecialCharTok{$}\NormalTok{Span)}
\NormalTok{logLength }\OtherTok{=} \FunctionTok{log10}\NormalTok{(aircraft}\SpecialCharTok{$}\NormalTok{Length)}
\NormalTok{logWeight }\OtherTok{=} \FunctionTok{log10}\NormalTok{(aircraft}\SpecialCharTok{$}\NormalTok{Weight)}
\NormalTok{logSpeed }\OtherTok{=} \FunctionTok{log10}\NormalTok{(aircraft}\SpecialCharTok{$}\NormalTok{Speed)}
\NormalTok{logRange }\OtherTok{=} \FunctionTok{log10}\NormalTok{(aircraft}\SpecialCharTok{$}\NormalTok{Range)}

\NormalTok{Period }\OtherTok{=} \FunctionTok{as.factor}\NormalTok{(aircraft}\SpecialCharTok{$}\NormalTok{Period)}

\NormalTok{Year}\OtherTok{=}\NormalTok{ aircraft}\SpecialCharTok{$}\NormalTok{Year}

\NormalTok{log\_aircraft }\OtherTok{=} \FunctionTok{data.frame}\NormalTok{(Year, Period, logPower, logSpan, logLength, logWeight, logSpeed, logRange)}

\NormalTok{aircraft.per1 }\OtherTok{=} \FunctionTok{filter}\NormalTok{( log\_aircraft, Period }\SpecialCharTok{==} \StringTok{"1"}\NormalTok{ )}
\NormalTok{aircraft.per2 }\OtherTok{=} \FunctionTok{filter}\NormalTok{( log\_aircraft, Period }\SpecialCharTok{==} \StringTok{"2"}\NormalTok{ )}
\NormalTok{aircraft.per3 }\OtherTok{=} \FunctionTok{filter}\NormalTok{( log\_aircraft, Period }\SpecialCharTok{==} \StringTok{"3"}\NormalTok{ )}

\CommentTok{\# For Period == 1 \#}
\FunctionTok{ggplot}\NormalTok{( aircraft.per1, }\FunctionTok{aes}\NormalTok{(logPower)  ) }\SpecialCharTok{+} \FunctionTok{geom\_density}\NormalTok{() }\SpecialCharTok{+} \FunctionTok{ggtitle}\NormalTok{(}\StringTok{"Period 1 logPower"}\NormalTok{)}
\end{Highlighting}
\end{Shaded}

\includegraphics{Assignment-1-Mikayil-Aliyev_files/figure-latex/unnamed-chunk-2-1.pdf}

\begin{Shaded}
\begin{Highlighting}[]
\FunctionTok{ggplot}\NormalTok{( aircraft.per1, }\FunctionTok{aes}\NormalTok{(logLength)  ) }\SpecialCharTok{+} \FunctionTok{geom\_density}\NormalTok{() }\SpecialCharTok{+} \FunctionTok{ggtitle}\NormalTok{(}\StringTok{"Period 1 logLength"}\NormalTok{)}
\end{Highlighting}
\end{Shaded}

\includegraphics{Assignment-1-Mikayil-Aliyev_files/figure-latex/unnamed-chunk-2-2.pdf}

\begin{Shaded}
\begin{Highlighting}[]
\CommentTok{\# For Period == 2 \#}
\FunctionTok{ggplot}\NormalTok{( aircraft.per2, }\FunctionTok{aes}\NormalTok{(logPower)  ) }\SpecialCharTok{+} \FunctionTok{geom\_density}\NormalTok{() }\SpecialCharTok{+} \FunctionTok{ggtitle}\NormalTok{(}\StringTok{"Period 2 logPower"}\NormalTok{)}
\end{Highlighting}
\end{Shaded}

\includegraphics{Assignment-1-Mikayil-Aliyev_files/figure-latex/unnamed-chunk-2-3.pdf}

\begin{Shaded}
\begin{Highlighting}[]
\FunctionTok{ggplot}\NormalTok{( aircraft.per2, }\FunctionTok{aes}\NormalTok{(logLength)  ) }\SpecialCharTok{+} \FunctionTok{geom\_density}\NormalTok{() }\SpecialCharTok{+} \FunctionTok{ggtitle}\NormalTok{(}\StringTok{"Period 2 logLength"}\NormalTok{)}
\end{Highlighting}
\end{Shaded}

\includegraphics{Assignment-1-Mikayil-Aliyev_files/figure-latex/unnamed-chunk-2-4.pdf}

\begin{Shaded}
\begin{Highlighting}[]
\CommentTok{\# For Period == 3 \#}
\FunctionTok{ggplot}\NormalTok{( aircraft.per3, }\FunctionTok{aes}\NormalTok{(logPower)  ) }\SpecialCharTok{+} \FunctionTok{geom\_density}\NormalTok{() }\SpecialCharTok{+} \FunctionTok{ggtitle}\NormalTok{(}\StringTok{"Period 3 logPower"}\NormalTok{)}
\end{Highlighting}
\end{Shaded}

\includegraphics{Assignment-1-Mikayil-Aliyev_files/figure-latex/unnamed-chunk-2-5.pdf}

\begin{Shaded}
\begin{Highlighting}[]
\FunctionTok{ggplot}\NormalTok{( aircraft.per3, }\FunctionTok{aes}\NormalTok{(logLength)  ) }\SpecialCharTok{+} \FunctionTok{geom\_density}\NormalTok{() }\SpecialCharTok{+} \FunctionTok{ggtitle}\NormalTok{(}\StringTok{"Period 3 logLength"}\NormalTok{)}
\end{Highlighting}
\end{Shaded}

\includegraphics{Assignment-1-Mikayil-Aliyev_files/figure-latex/unnamed-chunk-2-6.pdf}

\begin{enumerate}
\def\labelenumi{\alph{enumi})}
\setcounter{enumi}{1}
\tightlist
\item
  Construct contour plots of the 2D smoothed histograms of the pairs
  (logPower, logWeight) and (logSpeed, logLength). Describe the shapes
  of the density plots and discuss how they change over time.
\end{enumerate}

\begin{Shaded}
\begin{Highlighting}[]
\FunctionTok{ggplot}\NormalTok{( log\_aircraft, }\FunctionTok{aes}\NormalTok{( logPower, logWeight )) }\SpecialCharTok{+} \FunctionTok{geom\_density\_2d}\NormalTok{( ) }\SpecialCharTok{+} \FunctionTok{geom\_point}\NormalTok{( }\FunctionTok{aes}\NormalTok{( }\AttributeTok{colour =}\NormalTok{ Period))}
\end{Highlighting}
\end{Shaded}

\includegraphics{Assignment-1-Mikayil-Aliyev_files/figure-latex/unnamed-chunk-3-1.pdf}

\begin{Shaded}
\begin{Highlighting}[]
\FunctionTok{ggplot}\NormalTok{( log\_aircraft, }\FunctionTok{aes}\NormalTok{( logSpeed, logLength )) }\SpecialCharTok{+} \FunctionTok{geom\_density\_2d}\NormalTok{( ) }\SpecialCharTok{+} \FunctionTok{geom\_point}\NormalTok{( }\FunctionTok{aes}\NormalTok{( }\AttributeTok{colour =}\NormalTok{ Period))}
\end{Highlighting}
\end{Shaded}

\includegraphics{Assignment-1-Mikayil-Aliyev_files/figure-latex/unnamed-chunk-3-2.pdf}

\end{document}
